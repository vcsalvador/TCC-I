%
% Documento: Metodologia
%

\chapter{Metodologia}

O primeiro passo da execução do trabalho consistirá na consolidação do conceito de manutenibilidade em código CSS. Pretendemos alcançar este objetivo por meio de referências bibliográficas e de uma pesquisa do tipo \textit{survey} com desenvolvedores profissionais, com níveis de experiência variados. Esta pesquisa pretende identificar as propriedades da linguagem e as situações mais comuns que dificultam, ou facilitam, a manutenção de códigos CSS.

Após executada a pesquisa, as métricas serão consolidadas, propondo valores individuais para as características da linguagem e definindo um índice de qualidade a partir de um agrupamento destes valores. Dessa forma serão obtidas informações suficientes para construir uma ferramenta para validar as métricas definidas.

Utilizando de ferramentas automatizadas, serão identificadas as relações entre índice proposto e a quantidade de falhas identificadas em sistemas \textit{web} de código aberto. Analisando produções do GitHub, iremos fazer a referência cruzada, entre a pontuação alcançada pelo código CSS e o número de problemas reportados relacionados ao mesmo.

\section{\textit{Survey}}

A pesquisa \textit{survey} é uma forma de obtenção de dados ou informações sobre características, ações ou opiniões de um determinado conjunto de pessoas, indicado como representante de uma população-alvo, por meio de um instrumento de pesquisa, normalmente um questionário \cite{Freitas2000}. 

O interesse de uma pesquisa desse tipo é produzir descrições quantitativas de uma população. No caso deste trabalho, o objetivo é identificar, de forma quantitativa, as características identificadas pelos desenvolvedores como sendo as que classificam a legibilidade do código. Para tanto, será executado um \textit{survey} exploratório, com o objetivo de identificar os conceitos do CSS que são centrais para a associação de qualidade do código.

\section{Proposta das Métricas}

O objetivo desse trabalho é encontrar métricas de qualidade, com o intuito de melhorar a manutenibilidade do código, garantindo que o código é de fácil legibilidade, entendimento e evite os possíveis efeitos colaterais dessa manutenção.

As métricas são identificadores numéricos baseados em características da linguagem. No caso do CSS, essas características serão definidas a partir dos resultados do \textit{survey}.

\section{Avaliação dos resultados}

A partir da métrica proposta será desenvolvida uma ferramenta de validação automática. O programa irá ler o arquivo CSS, identificará as regras definidas e, a partir da definição proposta, será calculado para cada arquivo uma métrica. Se a métrica proposta identificar a necessidade, os arquivos HTML também serão considerados no cálculo da métrica.

Com as métricas calculadas, será testado a aderência da métrica a projetos \textit{opensource}, utilizando o número de problemas relacionados a CSS como parâmetro de triangulação. Dessa forma será construída uma base de dados para as análises estatísticas, que confrontaram os resultados esperados deste trabalho.
