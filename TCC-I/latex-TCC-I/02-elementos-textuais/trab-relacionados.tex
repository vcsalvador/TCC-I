%
% Documento: Trabalhos Relacionados
%

\chapter{Trabalhos Relacionados}

Este capítulo inclui muitas citações bibliográficas. Os principais
itens de bibliografia citados são livros, artigos em conferências,
artigos em {\textit journals} e páginas Web. A bibliografia deve seguir o
padrão ABNT\index{ABNT}\footnote{Este não é o endereço oficial da
ABNT pois as Normas Técnicas oficiais são pagas e não estão disponíveis na Web.}.

A bibliografia é feita no padrão {\ttfamily bibtex}.
As referências são colocadas em um arquivo separado.
Os elementos de cada item bibliográfico que devem constar na bibliografia são apresentados a seguir.

Para livros, o formato da bibliografia no arquivo fonte é o seguinte:

\begin{verbatim}
@Book{linked,
   author = {A. L. Barabasi},
   title = {Linked: The New Science of Networks},
   publisher = {Perseus Publishing},
   year = {2002},
}
\end{verbatim}

A citação deste livro se faz da seguinte forma \verb#\cite{linked}# e o resultado fica assim \cite{linked}.
Para os artigos em {\textit journals}, veja por exemplo \cite{acmsurveys},
descrito da seguinte forma no arquivo {\ttfamily .bib}:

\begin{verbatim}
@article{acmsurveys,
   author    = {Deepayan Chakrabarti and Christos Faloutsos},
   title     = {Graph mining: Laws, generators},
   journal   = {ACM Computing Surveys},
   volume    = {38},
   number    = {1},
   year      = {2006},
   pages     = {2-59},
   publisher = {ACM},
   address   = {New York, NY, USA},
}
\end{verbatim}

O artigo \cite{3faloutsos} foi publicado em conferência. Embora
às vezes seja difícil distinguir um artigo publicado em {\textit
 journal} de um artigo publicado em conferência, esta distinção é
fundamental. Em caso de dúvida, procure ajuda de seu orientador.

Veja também duas citações juntas \cite{rp99,mar00} e como citar
endereços Web \cite{irl:06}.
O trabalho realizado para editar as citações no formato correto é
compensado por uma bibliografia impecável.

\section{Citações livres}\label{citacoesLivres}
Citações são trechos transcritos ou informações retiradas das publicações consultadas para a realização do trabalho.
As citações são utilizadas no texto com o propósito de esclarecer, completar, embasar ou corroborar as ideias do autor.

Todas as publicações consultadas e efetivamente utilizadas (através de citações) devem ser listadas, obrigatoriamente, nas referências bibliográficas, de forma a preservar os direitos autorais e intelectuais.

Na utilização de citações, normalmente, utiliza-se referências.
Para cada tipo de referência presente no texto será apresentado um exemplo do comando utilizado para criá-lo.

Há basicamente dois tipos de citações: citações livres e citações literais.

Nas citações livres, reproduzem-se as ideias e informações de um autor, sem, entretanto, ``copiar letra por letra'' o texto do autor.
Há várias maneiras de se fazer uma citação livre, como mostra os exemplos abaixo.

Por outro lado, \citeonline{maturana:2003} defende um princípio de lógica.
Para o autor, quando dizemos \ldots

Além disso, \citeonline{teste:2004} argumenta que \ldots\mbox{ }Observe o detalhe do termo \textit{et al}.
que deve ser utilizado quando o trabalho citado possui mais de três autores.
Esse recurso é automatizado pelo estilo {\ttfamily abntex2}\index{ABNT!abntex2}.
Caso não haja desejo em abreviar o nome dos demais autores através do termo \textit{et al.}, deve-se incluir a opção {\ttfamily abnt-no-etal-label}.

Para evitar uma interrupção na sequência do texto, o que poderia, eventualmente, prejudicar a leitura, pode-se indicar a fonte entre parênteses imediatamente após a citação livre.
Porém, neste caso específico, o nome do autor deve vir em caixa alta, seguido do ano da publicação, como no exemplo a seguir.

A física, então, constituiu-se como a prova mínima da efetividade do método científico para descobrir as verdades do universo \cite{teste:2004,maturana:2003}.


\section{Citações literais}\label{citacoesLiterais}
Nas citações literais, reproduzem-se as ideias e informações de um autor, exatamente como este a expressou, ou seja, faz-se uma ``cópia letra por letra'' do texto do autor.
Há várias maneiras de se fazer uma citação literal, como mostra os exemplos abaixo.

As citações longas (mais de 3 linhas) devem usar um parágrafo específico para ela, na forma de um texto recuado (4 cm da margem esquerda), com tamanho de letra menor do aquela utilizada no texto e espaçamento simples entre as linhas, seguido dos sobrenomes dos autores em caixa alta (separados por ponto e vírgula), ano de publicação e número da página.
Veja o exemplo abaixo.

\begin{citacao}
Desse modo, opera-se uma ruptura decisiva entre a reflexividade filosófica, isto é a possibilidade do sujeito de pensar e de refletir, e a objetividade científica.
Encontramo-nos num ponto em que o conhecimento científico está sem consciência.
Sem consciência moral, sem consciência reflexiva e também subjetiva.
Cada vez mais o desenvolvimento extraordinário do conhecimento científico vai tornar menos praticável a própria possibilidade de reflexão do sujeito sobre a sua pesquisa \cite[p.~28]{morinmoigne:2000}.
\end{citacao}

Para se criar o efeito demonstrado na citação anterior, deve-se utilizar o comando:
\begin{verbatim}
    \begin{citacao}
        <citacao>
    \end{citacao}
\end{verbatim}

Opcionalmente, pode-se referenciar os autores no corpo de texto (neste caso seus nomes devem vir em minúsculas), e em seguida colocar a citação literal, em um novo parágrafo recuado.
Note que pode após a citação literal não mais aparece o nome dos autores, visto que já se encontra no texto.
Veja o exemplo seguinte.

\citeonline[p.~33]{morinmoigne:2000}, ao fazerem as suas críticas à ciência, explicitam uma ideia coletiva:

\begin{citacao}
Mas o curioso é que o conhecimento científico que descobriu os meios realmente extraordinários para, por exemplo, ver aquilo que se passa no nosso sol, para tentar conceber a estrutura das estrelas extremamente distantes, e até mesmo para tentar pesar o universo, o que é algo de extrema utilidade, o conhecimento científico que multiplicou seus meios de observação e de concepção do universo, dos objetos, está completamente cego, se quiser considerar-se apenas a si próprio!
\end{citacao}

As citações curtas (menos de 3 linhas) devem ser inseridas diretamente no texto (entre aspas), seguida do nome do autor (em caixa alta), ano e página, como no exemplo a seguir.

Então significa apenas que ``assumo que não posso fazer referência a entidades independentes de mim para construir meu explicar'' \cite[p.~35]{maturana:2003}.

O conhecimento de \citeonline[p.~35]{maturana:2003} aponta que isto significa apenas que ``assumo que não posso fazer referência a entidades independentes de mim para construir meu explicar''.

Finalmente, e isto vale para citações curtas ou longas, caso seja necessário inserir, no meio de uma citação uma palavra ou frase curta de sua autoria, que sirva para clarear ou completar a frase do autor citado, isto deve ser feito colocando a citação entre aspas.
O comentário deverá ser inserido sem aspas.
Ou seja, todo texto da citação deverá ficar envolvido por aspas.
O exemplo abaixo apresenta o resultado esperado.

Significa apenas que ``assumo que não posso fazer referência a entidades'' objetivas no sentido tradicional ``independentes de mim para construir meu explicar'' {\citeonline[p.~35]{maturana:2003}}.

\section{Informações sobre as referências utilizadas}\label{referenciasUtilizadas}

Nesta seção serão apresentadas os comandos necessários para a criação das referências utilizadas anteriormente.
As informações serão apresentadas da seguinte maneira:

\begin{itemize}
    \item \citeonline{maturana:2003}\\ \verb|\citeonline{maturana:2003}|
    \item \citeonline{teste:2004}\\ \verb|\citeonline{teste:2004}|
    \item \cite[p.~28]{morinmoigne:2000}\\ \verb|\cite[p.~28]{morinmoigne:2000}|
    \item \citeonline[p.~33]{morinmoigne:2000}\\ \verb|\citeonline[p.~33]{morinmoigne:2000}|
    \item \cite[p.~35]{maturana:2003}\\ \verb|\cite[p.~35]{maturana:2003}|
    \item \citeonline[p.~35]{maturana:2003}\\ \verb|\citeonline[p.~35]{maturana:2003}|
    \item \cite{teste:2004,maturana:2003}\\ \verb|\cite{teste:2004,maturana:2003}|
\end{itemize}
