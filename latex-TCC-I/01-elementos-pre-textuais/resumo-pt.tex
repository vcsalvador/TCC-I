%
% Documento: Resumo (Português)
%

\begin{resumo}

A qualidade de código CSS é uma área de conhecimento inexplorada. A codificação de folhas de estilo CSS não é uma tarefa trivial e para se chegar ao resultado desejado é necessário um certo esforço de tempo. Apesar de aparentar simplicidade, a linguagem CSS contém algumas armadilhas para os desenvolvedores, como o efeito cascata, herança de propriedades e especificidade. Devido a essa complexidade na codificação de CSS, a manutenção desse código se torna onerosa. Este trabalho identificou critérios de avaliação e pontos de interesse, em direção à uma métrica de qualidade, que viabilizam o estudo sobre a etapa de manutenção de códigos CSS. A partir de uma pesquisa exploratória foram identificados alguns aspectos da linguagem CSS que compõem a qualidade de código, do ponto de vista dos desenvolvedores, e a partir desses aspectos foi proposta uma métrica de qualidade para medir a manutenibilidade do código CSS. Para elaboração da métrica, foram definidos 12 critérios de avaliação, que representam algumas características da linguagem que apresentam dificuldade no momento de correção. Foi construído um calculador automático, desenvolvido em JavaScript, para avaliar o valor da métrica para qualquer página \textit{web}. Como objeto de estudo foi escolhido o Jenkins, por utilizar somente arquivos CSS para codificação de seu estilo e possuir um gerenciador de tarefas aberto, utilizando o JIRA, para identificação dos defeitos levantados ao longo de toda sua vida. A partir dos testes de versões do Jenkins, desde 2010 até a versão mais atual, foi calculada a métrica do seu código CSS ao longo do tempo, para comparar o comportamento da métrica em relação ao número de defeitos criados. Para verificar o comportamento da métrica em função das alterações feitas no código, foi escolhido o arquivo que sofreu maior número de \textit{commits}. Foi possível notar que a métrica foi crescente ao longo do tempo, o que indica que a longevidade de um código CSS o torna mais complexo de se manter. O comportamento da métrica em relação ao número de defeitos é característico do aumento da complexidade de manutenção, mas não determina a métrica como definitiva, serão necessárias mais iterações de testes e ajustes para a consolidação de uma métrica sólida. Mas este trabalho representa um passo em direção à métrica de qualidade de código CSS.

%Escrever código CSS não é uma tarefa trivial, visto que algumas características da linguagem representam. Devido a essas características, pode ser identificada uma série de fatores que dificultam a construção, manutenção e evolução do código CSS. A etapa de manutenção é essencial durante o tempo de vida do \textit{software}. Sendo assim, é necessário que seja encontrada uma forma de mitigar os possíveis impactos da modificação, ou evolução, do código CSS. Uma forma de diminuir os impactos na etapa de manutenção é criar uma métrica de qualidade que identifique o nível de manutenibilidade do código CSS. Essa necessidade ainda não foi suprida por nenhum trabalho acadêmico, portanto, é necessário ainda identificar quais são os aspectos que definem a qualidade do CSS. Para identificá-los, foi construído um questionário exploratório para fazer o levantamento das características do CSS que impactam na qualidade do código, visando uma forma de quantificar a manutenibilidade do código CSS. A partir da definição dos critérios de qualidade, foi proposta uma métrica para identificar a manutenibilidade do código e construído um \textit{script} para o cálculo automatizado dessa métrica, de forma a gerar dados para análise entre ela e o número de defeitos gerados a partir do código calculado. Através destes resultados é possível notar um progresso em direção à qualidade de código CSS. 

\textbf{Palavras-chave}: CSS. manutenibilidade. métrica. qualidade de código. Engenharia de Software. web.

\end{resumo}
