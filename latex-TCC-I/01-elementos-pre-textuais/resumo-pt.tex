%
% Documento: Resumo (Português)
%

\begin{resumo}

Escrever código CSS não é uma tarefa trivial, visto que algumas características da linguagem constantemente causam inconsistências arquiteturais que podem resultar em efeitos colaterais. Devido a essas características, podem ser identificados uma série de fatores que dificultam a construção, manutenção e evolução do código CSS. Estas etapas são essenciais durante o tempo de vida do \textit{software}, e isso não é diferente para aplicações \textit{web}. Sendo assim é necessário que seja encontrada uma forma de mitigar os possíveis impactos da modificação, ou evolução, do código CSS de um projeto \textit{web}. Uma forma de diminuir os impactos dessa etapa é criar uma métrica de qualidade que identifique o nível de manutenibilidade do código CSS. Essa necessidade ainda não foi suprida por nenhum trabalho acadêmico, portanto, é necessário ainda identificar quais são os aspectos que definem a qualidade do CSS. Para identificá-los, foi construído um questionário exploratório para fazer o levantamento das características do CSS que impactam na qualidade do código, visando uma forma de quantificar a manutenibilidade do código CSS. A partir da definição dos critérios de qualidade, foi construído um \textit{script} para o cálculo da métrica proposta, de forma a gerar dados para ela e o número de defeitos gerados a partir do código calculado. Através destes resultados é possível notar um progresso em direção à qualidade de código CSS. 

\textbf{Palavras-chave}: CSS. manutenibilidade. métrica. qualidade de código. Engenharia de Software. web.

\end{resumo}
