%
% Documento: Conclusão
%

\chapter{Conclusão}

Muitos propriedades de qualidade de código, como coesão e acoplamento, para linguagens orientadas a objeto, não se aplicam diretamente ao CSS. Mas isso não impede que seja traçada uma relação entre o CSS e essas propriedades.

Através do questionário, executado como piloto, notou-se alguns ajustes que deviam ser feitos para coletar com eficiência os dados necessários para proposição da métrica. Com o questionário publicado, espera-se conseguir um número mínimo de cinquenta respostas ao questionário, para que possa ser feita a análise das respostas e identificar as propriedades da linguagem.

\section{Próximos passos}

Após a coleta dos dados do questionário, será proposta a métrica, a partir de valores dos aspectos identificados como relacionados à qualidade de código fonte CSS. A análise dos dados será feita levando-se em conta o nível de conhecimento dos respondentes, como uma ponderação da variância da subjetividade da impressão de dificuldade.

Será construída uma ferramenta de cálculo automático da métrica, para o levantamento e execução dos testes de aderência com os projetos de código aberto avaliados. Para tal, será feita a comparação da pontuação da métrica de uma folha de estilo, com o número de problemas identificados na página \textit{web} relacionado à ela, buscando essas páginas em um repositório de código aberto. Ao final do trabalho, serão avaliados os resultados obtidos, para identificar a validade da métrica proposta.