%
% Documento: Conclusão
%

\chapter{Conclusão}

A qualidade de código CSS é uma área de conhecimento inexplorada. A partir dos resultados encontrados nesta pesquisa, é possível identificar um caminho a ser traçado para encontrar uma métrica definitiva.

%resumir o contexto, meu objetivo, metodologia, desenvolvimento e resultados

O questionário permitiu a identificação das características que compõem a qualidade de código CSS e as dificuldades encontradas durante a etapa de manutenção deste. Pode-se notar que a legibilidade e os efeitos colaterais são aspectos que exercem influência para a qualidade geral do código CSS. 

Os doze critérios de avaliação identificados, a partir do questionário, permitiram calcular o nível de manutenibilidade do código CSS da aplicação Jenkins, escolhida como objeto de estudo para validação da métrica, ao longo do tempo. A comparação dos resultados obtidos com o número de defeitos cadastrados no JIRA para cada versão do Jenkins testada, mostrou que o aumento do valor da métrica está relacionado com o aumento dos defeitos. Este fato demonstra que os parâmetros avaliados pelos critérios desenvolvidos estão relacionados com a qualidade geral do código.

A partir de uma análise do histórico de \textit{commits} dos arquivos CSS utilizados pela aplicação, foi feita a identificação do comportamento da qualidade em função das modificações feitas pela equipe no código. Foi então identificado que o arquivo CSS que mais sofreu alterações ao longo do tempo exibia o mesmo perfil de evolução da métrica, o que leva a entender que a participação da equipe de desenvolvimento teve impacto direto na qualidade do código CSS. %explicar isso aqui direito seu burro

Após as análises de impacto na qualidade dos resultados da métrica, identificou-se o perfil de composição dos critérios para o resultado final. Estas análises mostraram que as modificações ao longo do tempo impactaram no perfil de complexidade do código, indicados pelo surgimento de novos critérios de avaliação, ou aumento da participação total de outros.

O número de defeitos criados em uma aplicação não é a melhor forma de se determinar a manutenibilidade do código, mas pode ser um indicador do seu nível de qualidade. Portanto, os resultados encontrados não determinam uma métrica de manutenibilidade definitiva, mas iluminam o caminho para estudos futuros.

\section{Contribuições}

Este trabalho modifica o estado de inércia de uma área de estudo que não é muito abordada na comunidade científica, a qualidade de código CSS em função de sua manutenibilidade. Partindo do zero, conseguiu-se determinar doze critérios de avaliação para a métrica de qualidade, que foram capazes de demonstrar um comportamento factível da quantidade de defeitos encontrados no \textit{layout} de uma página \textit{web}. 

Durante a execução dos testes para identificação da métrica, construiu-se uma ferramenta de avaliação das folhas de estilo renderizadas em uma página \textit{web}. O \textit{script} construído é capaz de executar os testes dos critérios codificados e está disponível em um repositório aberto, para participação da comunidade e para utilização em trabalhos futuros.

Os resultados apresentados não formam uma métrica definitiva, mas representam um passo em direção à definição de qualidade de código CSS. Cabendo ainda a análise dos resultados sob a perspectiva de tempo em correções, ajustes nos pesos dos critérios e identificação de novos  para avaliação. Sendo assim, foi feita uma adição importante na área de Engenharia de Software, principalmente no que diz respeito à interface \textit{web}.

\section{Trabalhos Futuros}

Com o objetivo de dar continuidade a este trabalho, alguns pontos foram levantados como possíveis trabalhos futuros.

Devido à popularização do uso de pré-processadores CSS, vê-se como necessário a identificação de uma avaliação da métrica para este tipo de código.

Para melhor identificar os indicadores de qualidade e manutenibilidade do código, deve-se acompanhar todo o processo de desenvolvimento de uma aplicação \textit{web}, medindo o tempo gasto com a manutenção, evolução e correção do código CSS. Desta forma será possível identificar melhor o comportamento da métrica proposta com a etapa de manutenção do código.

Fazer modificações no arcabouço desenvolvido para que ele possa ser executado durante a fase de construção da aplicação, por exemplo, durante o processo de integração contínua, para que este processo não seja custoso à equipe de desenvolvimento.