\chapter{Introdução}
\label{chap:Intr}
A \textit{world wide web}, originalmente proposta como um meio para compartilhamento de documentos por Tim Berners-Lee (1989), torna-se cada vez mais popular. A \textit{web} é usada desde criação de páginas até sistemas de informação mais complexos, como comércio eletrônico, fóruns, clientes de email, portais de compartilhamento de vídeo etc \cite{Berners-Lee:2000:WWO:556560}.

Inicialmente proposta por Håkon Wium Lie e Bert Bos, a linguagem \textit{Cascading Style Sheet} (CSS) propunha a separação das páginas \textit{web} em um documento de conteúdo --- o arquivo HTML --- e um documento com a definição da aparência --- o documento CSS \cite{Hakon:2005}. O CSS se tornou largamente utilizado para definir a aparência e até mesmo certos comportamentos interativos em páginas \textit{web}, sendo um dos três padrões fundamentais da W3C\footnote{World Wide Web Consortium - http://www.w3.org/} para desenvolvimento de conteúdo \textit{web}, juntamente com o HTML e o Javascript.

Apesar das vantagens trazidas pela separação de responsabilidades, passou-se a gerar código CSS mais complexo e em maior quantidade, fazendo com que a sua manutenibilidade se tornasse mais onerosa quando de alterações corretivas ou evolutivas \cite{Mesbah2012}.
Escrever código CSS não é uma tarefa trivial, visto que algumas características da linguagem, como herança e especificidade de seletores, constantemente causam inconsistências arquiteturais que podem resultar em efeitos colaterais \cite{Walton:2015}.
Essas inconsistências podem prejudicar o que \citeonline{KellerNuss2010} definem como efetividade e eficiência de código CSS:

\begin{itemize}
	\item\textbf{Efetividade do código:} As propriedades de estilo, definidas no documento CSS, são efetivas se aplicadas aos elementos de conteúdo da forma desejada pelo desenvolvedor.
	
	\item\textbf{Eficiência do código:} Existem várias formas de se aplicar estilos aos elementos de conteúdo. Portanto, os códigos de CSS que podem ter o mesmo resultado ainda podem diferir significativamente. A eficiência de um código CSS significa implementar a atribuição de propriedades de forma a minimizar o esforço de codificar, manter e eventualmente reutilizar o código.
\end{itemize}

O efeito colateral descreve o fenômeno em que um agente que foi desenvolvido para afetar somente um escopo bem limitado acaba afetando um escopo muito maior. Folhas de estilo CSS têm escopo global e toda regra pode afetar partes desconexas do site, por isso efeitos colaterais são muito comuns. Uma vez que a folha de estilo, usualmente, consiste em uma coleção de regras altamente acopladas, totalmente dependentes na presença, ordem e especificidade de outras regras, até mesmo a menor mudança pode afetar a efetividade do código \cite{Walton:2015}.

\section{Motivação}

Devido às características da linguagem, pode-se identificar uma série de fatores que dificultam a construção, manutenção e evolução do código CSS. 

A manutenibilidade de um sistema é definida como a facilidade com a qual um \textit{software}, ou componente, pode ser modificado para corrigir falhas, melhorar performance, ou adaptar-se à mudança de ambiente \cite{Ieee1990}. Este trabalho pretende identificar, então, uma medida de manutenibilidade para código CSS.

A manutenção e modificação de um \textit{software} são etapas essenciais para o seu tempo de vida, e isso não é diferente para aplicações \textit{web}. Sendo uma tarefa essencial, e complexa, entende-se que seja necessário encontrar uma forma de mitigar os possíveis impactos na modificação, ou evolução, das folhas de estilo dos projetos \textit{web}.

As linguagens de folha de estilo, como o CSS, são muito pouco documentadas e pesquisadas \cite{Marden1999,Quint2007,Geneves2012}. E como identificado por \citeonline{Mesbah2012}, analisar código CSS sob uma perspectiva de manutenção ainda não foi explorado em nenhum trabalho científico. Portanto, há necessidade de se definir a qualidade do código CSS, com objetivo de se manter um nível de manutenibilidade da apresentação de páginas \textit{web}. Para medir esse nível, foi feita neste trabalho, uma proposta de métrica de manutenibilidade de código CSS, como forma de auxiliar no avanço de uma definição de qualidade.

\section{Objetivos}
Esta pesquisa pretende atender os seguintes objetivos específicos:

\begin{itemize}
	\item Identificar os aspectos da linguagem CSS que qualificam uma folha de estilo;
	\item Analisar os aspectos levantados e propor uma medida para a folha de estilo;
	\item Fazer uma análise e avaliar a relevância da métrica proposta a outros indicativos de manutenibilidade de código.
\end{itemize}

A partir desses objetivos, pretende-se propor uma métrica de manutenibilidade para códigos CSS, colaborando com a medição de folhas de estilo e auxiliando nos processos de produção para a plataforma \textit{web}.

%Esta pesquisa tem por objetivo propor uma métrica de manutenibilidade de código CSS. Para tanto, será feito um levantamento das propriedades da linguagem que modificam a facilidade da manutenção de seu produto, considerando a visão do autor e os aspectos funcionais.

%Pretende-se identificar, junto à comunidade desenvolvedora de CSS, os parâmetros fundamentais da qualidade de código CSS, bem como as principais dificuldades na etapa de manutenção de uma folha de estilo. Utilizando-se desses parâmetros, pretende-se criar um modelo de métrica para identificarmos quantitativamente a manutenibilidade de um código fonte, validando com uma verificação cruzada da quantidade de retrabalho, referente ao estilo, com o valor obtido pela métrica. 