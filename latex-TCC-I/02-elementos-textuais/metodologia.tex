%
% Documento: Metodologia
%

\chapter{Metodologia}

O primeiro passo da execução do trabalho consistirá na conceituação de manutenibilidade em código CSS. Pretende-se alcançar este objetivo por meio de referências bibliográficas e de uma pesquisa do tipo \textit{survey} aplicada a pessoas que escrevem código CSS em seu dia a dia, com níveis de experiência com a linguagem variados. Essa pesquisa pretende identificar as propriedades da linguagem e as situações mais comuns que dificultam, ou facilitam, a manutenção de código CSS.

Após executada a pesquisa, as métricas serão consolidadas, propondo valores individuais para as características da linguagem e definindo um índice de qualidade a partir de um agrupamento desses valores. Dessa forma, serão obtidas informações suficientes para construir uma ferramenta que calcule a métrica de um código CSS.

Utilizando de ferramentas automatizadas, serão identificadas as relações entre o índice proposto e a quantidade de falhas identificadas em sistemas \textit{web} de código aberto. Utilizando bases de projetos de \textit{software} de código aberto, será feita uma analise e a referência cruzada, entre o valor obtido pela métrica do código CSS e o número de problemas reportados relacionados ao projeto do mesmo.

\section{Questionário}

A pesquisa \textit{survey} é uma forma de obtenção de dados ou informações sobre características, ações ou opiniões de um determinado conjunto de pessoas, indicado como representante de uma população-alvo, por meio de um instrumento de pesquisa, normalmente um questionário \cite{Freitas2000}. 

O interesse de uma pesquisa desse tipo é produzir descrições quantitativas de uma população. No caso deste trabalho, o objetivo é identificar, de forma quantitativa, as características identificadas pelos desenvolvedores como sendo as que classificam a qualidade do código CSS. Para tanto, será aplicado um questionário exploratório, com o objetivo de identificar os conceitos do CSS que são centrais para a associação de qualidade do código.

\section{Proposta das Métricas}

As métricas são identificadores numéricos baseados em características da linguagem. No caso do CSS, estas características serão definidas a partir dos resultados obtidos no questionário.

\section{Avaliação dos resultados}

A partir da métrica proposta, será desenvolvida uma ferramenta automática de cálculo da métrica. O programa irá ler o arquivo CSS, identificar as regras definidas e, a partir da definição proposta, calcular o valor obtido por cada arquivo. Se for necessário para o calculo da métrica, os arquivos HTML também serão considerados no cálculo.

Com as métricas calculadas, será testada a aderência da métrica a alguns projetos de código aberto, utilizando o número de problemas relacionados a CSS --- como \textit{bugs} identificados nas bases de código aberto pelos colaboradores ---  como parâmetro de triangulação. Dessa forma, será construída uma base de dados para as análises estatísticas, que confrontarão os resultados esperados deste trabalho.
