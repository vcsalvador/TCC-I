%
% Documento: Resultados Esperados
%

\chapter{Resultados Esperados}

Entende-se que códigos inutilizados são o limite inferior da escala, uma vez que eles só ocuparam espaço no documento, dificultando a leitura e identificação de efeitos colaterais, se houverem. Por sua vez, propriedades que não sobrescrevam nenhuma outra, são o limite superior da escala. Muitas características da linguagem podem reduzir o valor dessas últimas, como especificidade do seletor, profundidade, localidade, etc.

Espera-se poder identificar de forma empírica quais são qualidades de manutenibilidade que atendam ao objetivo, de forma que nossos índices possam auxiliar na escrita de códigos que sejam modificados sem causar problemas adicionais.

Espera-se que a utilização das métricas propostas não influenciem de forma negativa em questões de desempenho, apesar de esta não ser uma característica vital da pesquisa aqui projetada.

Uma métrica válida de manutenção seria inédita na literatura, como identificado por \citeonline{Mesbah2012}. Este trabalho pretende encontrar uma norma que identifique a manutenibilidade das folhas de estilo, que será uma grande contribuição para a comunidade de desenvolvedores \textit{web} e \textit{designers}. 