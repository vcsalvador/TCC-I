%
% Documento: Trabalhos Relacionados
%

\chapter{Trabalhos Relacionados}

Existem poucos trabalhos que tratem de qualidade de código CSS na literatura, e como \citeonline{Mesbah2012} identificou, não existem trabalhos que analisem o código em função da manutenibilidade.

Existem trabalhos como o de \citeonline{KellerNuss2010}, que analisa a qualidade de código CSS em uma perspectiva de avaliar a diferença em códigos de autoria humana e os auto gerados. Enquanto \citeonline{Mesbah2012} propõem uma ferramenta para auxiliar na manutenção de código encontrando regras inefetivas e removendo-as do código.

\citeonline{KellerNuss2010} propõem uma medida de qualidade do código CSS baseando-se na abstração do seletor. Esse trabalho é baseado no argumento de que o objetivo do código CSS é a reutilização de suas regras, a abstração do seletor é então definida pela sua utilização no escopo geral do HTML, considerando que seletores com id são os menos abstratos possíveis. Este trabalho não conseguiu encontrar uma relação forte com a complexidade de código CSS e o nível de abstração, de forma que os autores a consideraram uma medida fraca, se utilizada de forma exclusiva, deixando em aberto a proposta de métricas que a corroborem, ou cooperem com ela na medida de qualidade do código CSS.

Em \citeonline{Quint2007} ferramentas de auxílio na autoria de CSS são avaliadas, e a forma de avaliação do código, como estrutura de dados e método de avaliação do código. Esse trabalho propõe uma ferramenta didática, que auxiliará na criação de folhas de estilo, auxiliando o entendimento da aplicação das propriedades e regras. Neste trabalho são também identificados fatores que tornam a tarefa de autoria de código CSS não trivial.
