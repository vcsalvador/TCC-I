%
% Documento: Trabalhos Relacionados
%

\chapter{Trabalhos Relacionados}

Antes de se falar em qualidade de código CSS, é bom entendermos do que se trata a qualidade de código fonte para os estudiosos de computação. 
A partir da perspectiva dos trabalhos relevantes para área de conhecimento da qualidade de código fonte, será possível argumentar com uma base sólida as hipóteses e resultados encontrados nessa pesquisa.

Neste capítulo serão abordado os trabalhos de grande relevância serão apontados e confrontados com o objetivo chave desse trabalho.

\section{Qualidade de Código Clássica}

A norma ISO 9126 define seis atributos-chave de qualidade de \textit{software} de computador, em que um dos atributos é a manutenibilidade \cite{Pressman:2010}. E a \citeonline{Ieee1990} define métrica como uma medida quantitativa do grau em que um sistema, componente ou processo possui um determinado atributo. Pode-se definir a medida quantitativa do grau de manutenibilidade de um código fonte como uma métrica.

Existem vários trabalhos na área de medição de \textit{software}, como por exemplo o trabalho de \citeonline{Whitmire:1997}, que discute os atributos chave que definem a qualidade de um sistema de \textit{software}, enquanto outros trabalhos exploram as medições de coesão, acoplamento e complexidade, que compõem os nove atributos chave \cite{McCabe:1989,Zuse:1991,Bieman1994,Dhama:1995,Zuse:1997} .

\citeonline{Riaz:2009} discutem os trabalhos relacionados a métricas e previsão de manutenibilidade de código fonte. Por meio de uma revisão bibliográfica, eles concluem que os modelos de previsão de manutenibilidade mais utilizados se baseiam em técnicas algorítmicas, sem encontrar distinção de qual modelo deve ser utilizado para cada sub característica ou tipo de manutenção.

\section{Qualidade de Código CSS}

Existem poucos trabalhos que abordam a qualidade de código CSS na literatura e, como \citeonline{Mesbah2012} identificaram, não existem trabalhos que analisem o código em função da sua manutenibilidade.

\citeonline{KellerNuss2010} analisam a qualidade de código CSS sob uma perspectiva de avaliar a diferença entre códigos de autoria humana e os gerados de forma automática. \citeonline{Mesbah2012} propõem uma ferramenta para auxiliar na manutenção de código, encontrando regras inefetivas e removendo-as do código.

\citeonline{KellerNuss2010} propõem uma medida de qualidade do código CSS baseando-se na abstração do seletor. Esse trabalho é baseado no argumento de que o objetivo do código CSS é a reutilização de suas regras. A abstração do seletor é então definida pela sua utilização no escopo geral de um documento HTML, considerando que seletores com id são os menos abstratos possíveis. O trabalho não conseguiu encontrar uma relação forte entre a complexidade de código CSS e o nível de abstração, de forma que os autores a consideraram uma medida fraca, se utilizada de forma exclusiva, deixando em aberto a proposta de métricas que a corroborem, ou cooperem com ela na medida de qualidade do código CSS.

\citeonline{Quint2007} analisam os requisitos necessários para construir uma ferramenta de manipulação de estilos, baseando-se nas estruturas principais da linguagem CSS. Além disso, também discutem os métodos e técnicas que podem atender a esses requisitos, auxiliando os autores \textit{web} de forma eficiente. O trabalho discute a necessidade das linguagens de estilo em documentos \textit{web} de uma forma geral, mas opta por focar no CSS, por ser mais utilizado e ter uma estrutura mais simples que o XSL\footnote{\textit{Extensible Stylesheet Language}}, por exemplo, que também possui alguns estudos de mesma natureza.

\citeonline{Park2013} investigam os erros cometidos pelas pessoas ao codificar HTML e CSS. Aplicando um método de análise, foi possível dividir em seguimentos as dificuldades enfrentadas e os erros cometidos pelos participantes da pesquisa. Os resultados encontrados demonstraram que é possível utilizar o \textit{framework skills-rules-knowledge} para análise de erros nos códigos, enquanto proveem uma compreensão da origem destes erros, e sugerem formas de se aprimorar ferramentas de desenvolvimento \textit{web} para o suporte ao aprendizado de HTML e CSS.

Verificou-se, através dos trabalhos citados, a carência da medição de qualidade do código CSS. Como exposto por \citeonline{Park2013}, o trabalho de codificar CSS não é trivial e necessita de suporte para o seu melhor desenvolvimento. Mas diferente do objetivo de \citeonline{KellerNuss2010}, pretende-se com esse trabalho identificar as maiores dificuldades na manutenção de CSS. 

Motivado pela escassez de trabalhos com esse objetivo, como explicitado por \citeonline{Mesbah2012}, elaborou-se uma métrica de qualidade para código CSS visando a sua manutenibilidade, um dos atributos-chave da ISO 9126. Utilizando o conceito de efeito colateral, exposto por \citeonline{Walton:2015}, como principal argumento para identificação de códigos de difícil manutenção.