\begin{quadro}[!htb]
	\centering
	\caption{Contribuição das respostas da questão 6 do questionário. (\autoref{chap:apendiceA})}
	\label{qd:openAnswers}
	\begin{tabular}{|l|c|c|}
		\hline
		\textbf{Resposta}                                                                                                                                                                                                                                                                                                                                                                                                                                                                                                            & \multicolumn{1}{l|}{\textbf{Corrobora}} & \multicolumn{1}{l|}{\textbf{Refuta}} \\ \hline
		Estilizar elementos sem classe, criar folhas de estilos muito extensas.                                                                                                                                                                                                                                                                                                                                                                                                                                                      & h1                                      &                                      \\ \hline
		CSS's que são atribuídos de forma mais genérica aos elementos.                                                                                                                                                                                                                                                                                                                                                                                                                                                                &                                         & h2                                   \\ \hline
		\begin{tabular}[c]{@{}l@{}}Regras complexas \\ Herança de valor de propriedade (valores,inherit, initial)\\ Aninhamento (seleção de elementos aninhados)\end{tabular}                                                                                                                                                                                                                                                                                                                                                  & h5                                      &                                      \\ \hline
		\begin{tabular}[c]{@{}l@{}}Utilização de nomes muitos genéricos para classes ou atributos.\\ A estilização que não é mais usada e fica no código.\end{tabular}                                                                                                                                                                                                                                                                                                                                                               & h3;h6                                   &                                      \\ \hline
		\begin{tabular}[c]{@{}l@{}}Regras para itens muito genéricos. \\ Utilização de !imporant. \\ Código repetido.\end{tabular}                                                                                                                                                                                                                                                                                                                                                                                                   & h6                                      & h2                                   \\ \hline
		Saber se um seletor está ou não sendo usado em algum parte do código.                                                                                                                                                                                                                                                                                                                                                                                                                                                        & h6                                      &                                      \\ \hline
		Seletores muito específicos.                                                                                                                                                                                                                                                                                                                                                                                                                                                                                                  & h2                                      &                                      \\ \hline
		\begin{tabular}[c]{@{}l@{}}Regras com seletores muito gerais, como classes e tags, \\ costumam provocar efeitos colaterais com mais frequência.\\ Acho que para poder escrever regras desse tipo (gerais), \\ todas as propriedades sendo definidas precisam ser "óbvias" \\ (fácil de uma 3ª pessoa entender por que ela está ali) \\ e também gerais (não sendo algo como uma classe .button \\ definindo um left:54px, que deveria estar sendo aplicado \\ a apenas um .button em particular e não a todos).\end{tabular} & h2                                      &                                      \\ \hline
		\begin{tabular}[c]{@{}l@{}}Arquivo desorganizado, regras repetidas, \\ sem sessões definidas,código compactado.\end{tabular}                                                                                                                                                                                                                                                                                                                                                                                                  & h3;h6                                   &                                      \\ \hline
	\end{tabular}
	\fonte{Próprio autor, a partir de respostas do questionário}
\end{quadro}