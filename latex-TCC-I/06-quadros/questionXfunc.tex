\begin{quadro}[!htb]
\centering
\caption{Quadro com as funcionalidades exploradas em cada questão do questionário no \autoref{chap:apendiceA}}
\label{quad:questionXfunc}
\begin{tabular}{|c|l|}
\hline
\textbf{Questão} & \textbf{Funcionalidade}                                                  \\ \hline
Q1               & Seletor de classe simples                                                \\ \hline
Q2               & Seletor de id simples                                                    \\ \hline
Q3               & Pseudo seletor de substring (\textasciicircum =)                         \\ \hline
Q4               & Agrupamento de seletores                                                 \\ \hline
Q5               & Pseudo classe                                                            \\ \hline
Q6               & Propriedades simplificadas                                               \\ \hline
Q7               & Pseudo elemento                                                          \\ \hline
Q8               & Seletor muito longo                                                      \\ \hline
Q9               & At-rule                                                                  \\ \hline
Q10              & Media queries                                                            \\ \hline
Q11              & Seletores com webkit, utilizando propiedades de webkit                   \\ \hline
Q12              & Uso da clausula not                                                      \\ \hline
Q13              & Child selector                                                           \\ \hline
Q14              & Seletor de alta complexidade: localização + sibling  + seletor universal \\ \hline
Q15              & Conflito de herança na cor da tag b                                      \\ \hline
Q16              & Seletor de localidade: first-child                                       \\ \hline
\end{tabular}
\end{quadro}